\documentclass[Afour,sageh,times]{sagej}
 %DRAFT MODE preserves formatting otherwise; %comment out
% \usepackage[draft]{graphics}

\let\labelindent\relax
\input{preamble}



\newenvironment{phase}[1][htb]
  {\renewcommand{\algorithmcfname}{Algorithm}% Update algorithm name
   \begin{algorithm}[#1]%
  }{\end{algorithm}}


% Macro for Algorithm Name 
\newcommand{\hirl}{SWIRL\xspace}
\newcommand{\HIRL}{SWIRL\xspace}
\newcommand{\hirlfull}{Sequential Windowed Inverse Reinforcement Learning\xspace}
\newcommand{\tsh}{TSH\xspace}
\newcommand{\TSH}{TSH\xspace}
\newcommand{\tshfull}{Transition State Hashing\xspace}
\newcommand{\tsco}{TSC\xspace}
\newcommand{\sys}{\textsf{TSC+VIS}\xspace}

\newcommand{\tsce}{\texttt{TSC-Endpoint}\xspace}
\newcommand{\tsci}{\texttt{TSC-IRL}\xspace}

\newboolean{include-notes}
\setboolean{include-notes}{true}
\newcommand{\fp}[1]{\ifthenelse{\boolean{include-notes}}%
 {\textcolor{blue}{\textbf{FP: #1}}}{}}
%===============================================================

\renewcommand\bibsection
  {\section*{\refname}\small\renewcommand\bibnumfmt[1]{##1.}}
 

\title{\LARGE \bf \hirl: A \hirlfull Algorithm for Robot Tasks With Delayed Rewards}

\author{%
Sanjay Krishnan, 
Animesh Garg, 
Richard Liaw,
Brijen Thananjeyan,\\
Lauren Miller,
Florian T. Pokorny$^{*}$,
Ken Goldberg%, 
}

\affiliation{\affilnum{1}AUTOLAB UC Berkeley\\
\affilnum{2}KTH Sweden}

\corrauth{Sanjay Krishnan}


%\IEEEoverridecommandlockouts %to enable thanks to appear

% ----------------------------------------------------------
% --- SPACING OVERRIDES ---
% ----------------------------------------------------------

% \onehalfspacing
\selectfont

%fix spacing after floats
% \setlength{\textfloatsep}{8pt}% Remove \textfloatsep


% ----------------------------------------------------------
\begin{document}

\begin{abstract}
\textbf{SK. TODO revise}
Inverse Reinforcement Learning (IRL) allows a robot to generalize from demonstrations to previously unseen scenarios by learning the demonstrator's reward function.
However, in multi-step tasks, the learned rewards might be delayed and hard to directly optimize.
We present \hirlfull (\hirl), a three-phase algorithm that partitions a complex task into shorter-horizon subtasks that occur consistently across demonstrations.
\hirl then learns a sequence of local reward functions that describe the motion between transitions. 
Once these reward functions are learned, \hirl applies Q-learning to compute a policy that maximizes the rewards. 
We formalize requirements on the segmentation algorithm that make this possible and evaluate \hirl in a number of discrete and continuous RL domains.
In physical experiments using the da Vinci surgical robot, we evaluate the extent to which \hirl generalizes from linear cutting demonstrations to cutting sequences of curved paths and can help learn multi-step policies for deformable tensioning.
\end{abstract} 

\maketitle



\section{Introduction}
One of the goals of learning from demonstrations (LfD) is to learn policies that generalize beyond the provided examples and are robust to perturbations in initial conditions, the environment, and sensing noise~\cite{argall2009survey}.
Inverse Reinforcement Learning (IRL) is a popular framework, where the objective is to infer an unknown reward function from a set of demonstrations~\cite{DBLP:conf/nips/KolterAN07, coates2008learning, abbeel2004apprenticeship}.
Once a reward is learned, given novel instances of a task, a policy can be computed by optimizing for this reward function using an policy search approach like Reinforcement Learning (RL)~\cite{ng2000algorithms, abbeel2004apprenticeship}.

In IRL, a task is modeled as an MDP with a single unknown function that maps states and actions to scalar values. 
This model is limited in the way that it can represent \emph{sequential tasks}, tasks where a robot must reach a sequence of intermediate state-space goals in a particular order.
In such tasks, the inferred reward may be \emph{delayed}
and reflect a quantity observed after all of the goals are reached, and thus, making it very difficult to optimize directly. 
Furthermore, there may not exist a single stationary policy for a given state-space that achieves all of the goals in sequence, e.g., a figure-8 trajectory in the x,y plane.

To address this problem, one approach is to divide the task into segments with local reward functions that ``build up'' to the final goal.
In existing work on multi-step IRL, this sequential structure is defined manually~\cite{DBLP:conf/nips/KolterAN07}.
We propose an approach that automatically learns sequential structure and assigns local reward functions to segments.
The combined problem is nontrivial because solving $k$ independent problems neglects the shared structure in the value function during the policy learning phase (e.g., a common failure state).
However, jointly optimizing over the segmented problem inherently introduces a dependence on history, namely, any policy must complete step $i$ before step $i+1$.
This potentially leads to an exponential overhead of additional states. 

\hirlfull (\hirl) is based on a model for sequential tasks that represents them as a sequence of reward functions $\mathbf{R}_{seq}=[R_1,...,R_k]$ and transition regions (subsets of the state-space) $G = [\rho_1, ...,\rho_k]$ such that $R_1$ is the reward function until $\rho_1$ is reached, after which $R_2$ becomes the reward and so on.
\hirl assumes that demonstrations are locally optimal (as in IRL), and all demonstrations reach each $\rho \in G$ in the same sequence.
In the first phase of the algorithm, \hirl segments the demonstrations and infers the transition regions using a kernelized variant of an algorithm proposed in our prior work~\cite{krishnan2015tsc,murali2016}.
In the second phase, \hirl uses the inferred transition regions to segment the set of demonstrations and applies IRL locally to each segment to construct the sequence of reward functions $\mathbf{R}_{seq}$.
Once these rewards are learned, \hirl computes a policy using an RL algorithm (Q-Learning) over an augmented state-space that indicates the sequence of previously reached reward transition regions. We show that this augmentation has an additional space complexity independent of the state-space and linear in the number of rewards.

\vspace{3pt}
\noindent \textbf{Our contributions are:}
\begin{enumerate}[
    topsep=0pt,
    noitemsep,
    % partopsep=1ex,
    % parsep=1ex,
    leftmargin=*,
    % itemindent=3ex
    ]
\item A three-phase algorithm, \hirl, to learn policies for sequential robot tasks.
\item We extend the Transition State Clustering algorithm~\cite{krishnan2015tsc,murali2016} with kernelization for segmentation that is robust to local non-linear (but smooth) dynamics.
\item A novel state-space augmentation to enforce sequential dependencies using binary indicators of the previously completed segments, which can be efficiently stored and computed based on the first phase of \hirl.
\item Simulation and physical experiments comparing \hirl with Supervised Learning and MaxEnt-IRL.
% Experiments suggest that \hirl can generalize from the demonstrations to unseen examples and is robust to noise in the environment and initial conditions. 
\end{enumerate}




%\todo{Add numbers}

%\documentclass[0-main.tex]{subfiles}
%\begin{document}
\section{Related Work}
The seminal work of Abbeel and Ng~\cite{abbeel2004apprenticeship} explored learning from demonstrations using Inverse Reinforcement Learning.
In~\cite{abbeel2004apprenticeship}, the authors used an IRL algorithm to infer the demonstrator's reward function and then an RL algorithm to optimize that reward.
Our work re-visits this two-phase algorithm in the context of sequential tasks.
It is well-established that RL problems often converge slowly in complex tasks when rewards are sparse and not ``shaped'' appropriately~\cite{DBLP:conf/icml/NgHR99, DBLP:conf/aaai/JudahFTG14}.
These issues are exacerbated in sequential tasks where a sequence of goals must be reached.
Related to this problem, Kolter et al. studied  \emph{Hierarchical Apprenticeship Learning} to learn bipedal locomotion~\cite{DBLP:conf/nips/KolterAN07}, where the algorithm is provided with a hierarchy sub-tasks.
These sub-tasks are not learned from data and assumed as given, but the algorithm infers a reward function from demonstrations.
\hirl applies to a restricted class of tasks defined by a sequence of reward functions and state-space goals.

There are have been some proposals in robotics to learn motion primitives from data. The approaches assume that reward functions are given (or the problem can be solved with planning-based methods).
Motion primitives are example trajectories (or sub-trajectories) that bias search in planning towards paths constructed with these primitives~\cite{ijspreet2002learning,pastor2009learning,manschitz2015learning}.
Much of the initial work in motion primitives considered manually identified segments, but recently, Niekum et al. \cite{niekum2012learning} proposed learning the set of primitives from demonstrations using the Beta-Process Autoregressive Hidden Markov Model (BP-AR-HMM).
Calinon et al.~\cite{calinon2014skills} proposed the task-parametrized movement model with GMMs for action segmentation.
Both Niekum and Calinon consider the motion planning setting in which analytical planning methods are used to solve a task and not RL.
Konidaris et al. studied the primitives in the RL setting~\cite{konidaris2011robot}. 
However, this approach assumed that the reward function was given and not learned from demonstrations as in \hirl.
Another relevant result is from Ranchod et al.~\cite{ranchod2015nonparametric}, who use an IRL model to define the primitives, in contrast to the problem of learning a policy after IRL.
\section{Problem Statement and Model}
\seclabel{back}

\subsection{Notation}
Consider a finite-horizon Markov Decision Process (MDP): \[\mathcal{M} = \langle S,A,P(\cdot,\cdot),\mathbf{R},T \rangle,\] where $S$ is the set of states (continuous or discrete), $A$ is the set of actions (finite and discrete), $P: S \times A \mapsto Pr(S)$ is the dynamics model that maps states and actions to a probability density over subsequent states, $T$ is the time-horizon, and $\mathbf{R}$ is a reward function that maps trajectories of length $T$ to scalar values.

Sequential tasks are tasks composed of sequences of sub-tasks. There is a sequence $\mathbf{R}_{seq}=[R_1,...,R_k]$, where each $R_i: S \times A \mapsto \mathbb{R}$. Associated with each $R_i$ is a transition region $\rho_i \subseteq S$. 
Each trajectory accumulates a reward $R_i$ until it reaches the transition $\rho_i$, then the robot switches to the next reward and transition pair.
This process continues until $\rho_k$ is reached.
A robot is deemed \emph{successful} when all of the $\rho_i \in G$ are reached in sequence.
Further, a robot is \emph{optimal} when it maximizes the expected cumulative reward and is successful.
Given observations of an optimally acting robot through a set of demonstration trajectories $D = \{d_1,...,d_k\}$, can we infer $\mathbf{R}_{seq}$ and $G$?

\vspace{0.5em}\noindent\textbf{Assumptions: } We make the following assumptions: (1) the changes in reward between transitions is non trivial (see next remark), (2) every demonstration is generated from $k$ distinct stationary, locally optimal policies (maximized w.r.t  $R_i$ on the infinite horizon), (3) every demonstration visits each $\rho_i$ in the same sequence, and (4) each $R_i$ is a quadratic of the form.
\vspace{0.5em}

\noindent \textbf{Remarks: } The key challenge in this problem is determining when a transition occurs--identifying the points in time in each trajectory at which the robot reaches a $\rho_i$ and transitions the reward function.
The natural first question is whether this is identifiable, that is, whether it is even theoretically possible to determine whether a transition $\rho_i \rightarrow \rho_{i+1}$ has occurred after obtaining an infinite number of observations. Trivially, this is not guaranteed when $R_{i+1} = R_{i}$, where it would be impossible to identify a transition purely from the robot's behavior (i.e., no change in reward, implies no change in behavior). Perhaps surprisingly, this is still not guaranteed even if $R_{i+1} \ne R_{i}$ due to policy invariance classes~\cite{DBLP:conf/icml/NgHR99}. Consider a reward function $R_{i+1} = 2R_{i}$, which functionally induce the same optimal behavior. Therefore, we consider a setting where all of the rewards in $\mathbf{R}_{seq}$ are distinct and are not equivalent w.r.t optimal policies.
This formalism is a special case of the Hierarchical Reinforcement Learning~\cite{dietterich2000hierarchical}, where each of the local rewards is a sub-goal and arrival at a $\rho$ is a termination condition. 

%The next challenge in the sequential task model is implicit partial observation. For sequential tasks, even if the each of the sub-tasks are fully observed the global problem might be partially observed. This is because the constraint that the robot must reach each of the $\rho_i$ in a particular sequence requires knowledge of the past (i.e., the set of previously reached transitions). So any inference procedure for $G$ and $R_{seq}$ must ensure that the event the robot has reached a given $\rho_i$ is a testable condition that only depends on prior observations. This problem is illustrated in Figure \ref{exp:gweasy1} in the experiments.

\subsection{Target Tracking Controllers}
As motivation for where such assumptions arise, consider the following system:
\[
x_{t+1} = Ax_{t} + Bu_{t} + w_{t}~~~~~w_{t} \sim N(0, \Sigma)~~~i.i.d 
\]
For quadratic rewards in the infinite horizon, the optimal policy is a linear state feedback controller $u_{t} = - C x_{t}$
Given this model, suppose we wanted to control the robot to a final state $\mu_i$ with a linear state-feedback controller $C_i$, the dynamical system that would follow is:
\[
\hat{x}_{t+1} = (A - B C_i) \hat{x}_t + w_t,
\]
where $\hat{x}_{t}  = x_t - \mu_i$. 
If this system is stable, it will converge to $x_{t} = \mu_i$ as $t \rightarrow \infty$.
Now, suppose that the system has the following switching behavior: when $\| x_{t} - \mu_i \| \le \epsilon$, change the target state $\mu_i$ to $\mu_{i+1}$.
The resulting closed loop dynamics are:
\[
A_i = (A - B C_i)
\]
\[
x_{t+1} = A_{i}\mathbf{x}_t + w_{t} \text{ : } A_i \in \{A_1,...,A_k\}.
\]
The equation above defines an SLDS.
This model maps back to the general case where the sequence $[\mu_1,...,\mu_k]$ and their tolerances $[\epsilon_1,...,\epsilon_k]$ define the regions $[\rho_1,...,\rho_k]$.
Each $\rho_i$ corresponds to regions where transitions occur $A_i \ne A_j$.
Intuitively, a change in the reward function results in a change of policy ($C_i$) for a locally optimal agent.

This form of a target-tracking linear system inspires our approach, where a robot applies a closed-loop controller to reach a target within a tolerance.
While the ultimate target is fixed, we still need to understand what cost function the robot is minimizing, i.e., are there directions in which the robot minimizes distance to the target faster? 
To learn these, we can leverage human demonstrations.
A demonstration $d$ is a trajectory of state and action tuples $[(s_0,a_0),...,(s_T,a_T)]$. 
Let $D$ be a set of demonstrations $\{d_1,...,d_N\}$, and our objective is to infer a sequence of reward functions (or cost functions) corresponding to each of the task segments.










\section{\hirlfull}
\seclabel{hirl}
This section describes an algorithm to infer the parameters for the proposed model.

\subsubsection{Algorithm Description}
Let $D$ be a set of demonstration trajectories $\{d_1,...,d_N\}$ of a task with a delayed reward.
\hirl can be described in terms of three sub-algorithms:

\vspace{2pt}
\noindent\textbf{Inputs:} Demonstrations $D$
\begin{enumerate}[
    topsep=0pt,
    noitemsep,
    % partopsep=1ex,
    % parsep=1ex,
    leftmargin=*,
    % itemindent=3ex
    ]
    \item \textbf{Sequence Learning: } Given $D$, \hirl segments the task into $k$ sub-tasks whose start and end are defined by arrival at a set of transitions $G = [\rho_1,...,\rho_k]$.
    \item \textbf{Reward Learning: } Given $G$ and $D$, \hirl associates a local reward function with the segment resulting in a sequence of rewards $\mathbf{R}_{seq}$. 
    \item \textbf{Policy Learning: } Given $\mathbf{R}_{seq}$ and $G$, \hirl applies reinforcement learning for $I$ iterations to learn a policy for the task $\pi$.
\end{enumerate}

\noindent\textbf{Outputs:} Policy $\pi$


\subsection*{Phase I. Sequence Learning}
\seclabel{algo}
The first phase of \hirl is to segment the demonstrations into locally linear segments.
We can then cluster the segment endpoints into $k$ clusters to infer  $[\rho_1,...,\rho_k]$. This is an extension of our prior work on robust task segmentation~\cite{krishnan2015tsc,murali2016}.
The overall procedure is summarized in Phase I.

\subsubsection{Segmentation and Transition Clustering}
The first step is given a set of demonstration trajectories, decompose each trajectory into segments.
A popular approach for segmentation is to use Gaussian Mixture Models~\cite{calinon2014skills}, namely, cluster all state observations and identify times at which $x_t$ is in a different cluster than $x_{t+1}$.
For a given time $t$, we can define a window of length $\ell$ as:
\[
\mathbf{n}^{(\ell)}_t = [x_{t-\ell},...,x_{t}]^\intercal
\]
Then, for each demonstration trajectory we can also generate a trajectory of $T_i - \ell$ windowed states:
\[
\mathbf{d}^{(\ell)}_i = [\mathbf{n}^{(\ell)}_\ell,...,\mathbf{n}^{(\ell)}_{T_i}]
\]
Over the entire set of windowed demonstrations, we collect a dataset of all of the $\mathbf{n}^{(\ell)}_t$ vectors.
We fit GMM model to these vectors.
The GMM model defines $m$ multivariate Gaussian distributions and a probability that each observation $\mathbf{n}^{(\ell)}_t$ is sampled from each of the $m$ distributions.
We annotate each observation with the most likely mixture component.
Times such that $\mathbf{n}^{(\ell)}_t$ and $\mathbf{n}^{(\ell)}_{t+1}$ have different most likely components are marked as transitions.
This has the interpretation of fitting a locally linear regression to the data (refer to~\cite{moldovan2013dirichlet, khansari2011learning, kruger2010learning, krishnan2015tsc,murali2016} for details).
In typical GMM formulations, one must specify the number of mixture components $k$ before hand.
However, we apply results from Bayesian non-parametric statistics and jointly solve for the component parameters and the number of components using a Dirichlet Process~\cite{kulis2011revisiting}.
Using a DP, the number of components grows with the complexity of the observed data (we denote this as DP-GMM).


In prior work, we noticed that motion-based segmentation algorithms can be unreliable when there is noise~\cite{krishnan2015tsc}.
We, however, realized that applying a second level of cluster--clustering the segment endpoints found dense clusters of common transitions that occurred in all demonstrations--thus allowing us to reject spurious motions or observation noise.
The insight of this work is that the same approach can be interpreted as identifying necessary sub-goals in a complex task.

We would like to be able to aggregate the transition times into state-space conditions for reward transitions $[\rho_1,...,\rho_k]$.
To each of these transition times, there is a corresponding \emph{transition state}--the last state before the dynamics switched.
We can model these regions againwith a Gaussian Mixture Model with $k$ mixture components $\{m_1,...,m_k\}$.
As before, we use a DP to non-parametrically set $k$.
Then, we prune clusters that do not have at least one transition from each demonstration.
Thus, the result is the set of transition regions: $G = [\rho_1, \rho_2,...,\rho_k]$, and segmentation of each demonstration trajectory into $k$ segments.

\subsubsection{Relaxing Local Linearity}
GMM's are a type of local Bayesian linear regression, but we can easily relax the linearity assumption.
We relax the linear dynamics assumption with a kernel embedding of the trajectories.
\hirl does not require learning the exact regimes $A_i$,  it only needs to detect changes in dynamics regime.
The basic idea is to apply Kernelized PCA to the features before learning the transitions--a technique used in Computer Vision~\cite{DBLP:conf/nips/MikaSSMSR98}.
By changing the kernel function (i.e., the similarity metric between states), we can essentially change the definition of local linearity.

Let $\mathbf{\kappa}(x_i,x_j)$ define a kernel function over the states.
For example, if $\mathbf{\kappa}$ is the radial basis function (RBF), then:
$ \mathbf{\kappa}(x_i,x_j) = e^{\frac{-\|x_i-x_j\|_2^2}{2\sigma}}$.
$\mathbf{\kappa}$ naturally defines a matrix $M$ where: $M_{ij} = \mathbf{\kappa}(x_i,x_j)$. 
The top $p'$ eigenvalues define a new embedded feature vector for each $\omega$ in $\mathbb{R}^{p'}$.
We can now apply the algorithm above in this embedded feature space.

\begin{phase}[t]
\small
\DontPrintSemicolon
\caption{Sequence Learning \label{alg:tsh1}}
\KwData{Demonstration $\mathcal{D}$}

Fit a DP-GMM model to $\mathcal{D}$ and identify the set of transitions $\Theta$, defined as all $(x_t,t)$  where  $(x_{t+1},t+1)$  has a different cluster.

Fit a DP-GMM to the states in $\Theta$.

Prune clusters that do not have one transition from all demonstrations.

The result of is $G = [\rho_1, \rho_2,...,\rho_m]$ where each $\rho$ is a disjoint ellipsoidal region of the state-space and time interval.

\KwResult{G}
\end{phase}

\section*{\hirl: Reward Learning}\label{sec:reward}
After the sequence learning phase, each demonstration is segmented into $k$ segments.
The reward learning phase uses the learned $[\rho_1,...,\rho_k]$ to construct the local rewards $[R_1,...,R_k]$ for the task.
Each $R_i$ is a quadratic cost parametrized by a positive semi-definite matrix $Q$.
The Algorithm is summarized in Phase\,\ref{alg:tsh2}.

\subsection{Reward Shaping with IRL}



\subsubsection{Model-based}
For the model-based approach, we use Maximum Entropy Inverse Reinforcement Learning (MaxEnt-IRL)~\cite{DBLP:conf/aaai/ZiebartMBD08}. 
The idea is to model every demonstration $d_i$ as a noisy sample from an optimal policy.
In other words, each $d_i$ that is observed is a noisy observation of some hypothetical $d^*$.

Since each $d_i$ is a path through a possibly discrete state and action space, we cannot simply average them to find $d^*$.
Instead, we have to model trajectories that the system is likely to visit. 
This can be formalized with the following probability distribution:
\[
P(d_i | R) \propto \exp \{ \sum_{t=0}^T R(s_t) \}.
\]
Paths with a higher cumulative reward are more likely.

MaxEnt-IRL uses the following linear parametrized representation:
\[
R(s,a) = x^T \theta,
\]
where $x$ is the state vector. The resulting form is:
\[
P(d_i | R) \propto \exp \{ \sum_{t=0}^T x^T \theta \},
\]
and MaxEnt-IRL proposes an algorithm to infer the $\theta$ that maximizes the posterior likelihood.

\hirl applies MaxEnt-IRL to each segment of the task but with a small modification to learn quadratic rewards instead of linear ones. 
Let $\mu_i$ be the centroid of the next transition region.
We want to learn a reward function of the form:
\[
R_i(x) = -(x-\mu_i)^T Q (x-\mu_i).
\]
for a positive semi-definite $Q$ (negated since this is a negative quadratic cost).
With some re-parametrization, this reward function can be written as:
\[
R_i(x) = -\sum_{j=1}^d \sum_{l=1}^d q_{ij} x[j] x[l].
\]
which is linear in the feature-space $y = x[j] x[l]$:
\[
R_i(x) = \theta^T y.
\]

This posterior inference procedure requires a dynamics model. We fit local linear models to each of the segments discovered in the previous section:
\[
A_j = \arg\min_{A} \sum_{i=1}^N \sum_{\text{seg j start}}^{\text{seg j end}} \|A x^{(i)}_{t} - x^{(i)}_{t+1}\|
\]
In this form, the problem can be analytically solved with techniques proposed in~\cite{ziebart2012probabilistic}.
\hirl applies MaxEnt-IRL to the sub-sequences of demonstrations between 0 and $\rho_1$, and then from $\rho_1$ to $\rho_2$ and so on.
The result is an estimated local reward function $R_{i}$ modeled as a linear function of states that is associated with each $\rho_i$.

\iffalse
\subsubsection{Application Scenarios} While typically RL is applied in scenarios where dynamics models are unavailable, there are scenarios in which the model-based reward approach applies. Suppose we are given demonstrations of a known system, e.g., a simulator or a simplified version of a task, with different dynamics $P'$ but with the same overall task structure.
While a policy learned in this setting would not transfer due to a change in dynamics, the reward function would.
In our experiments, we demonstrate an example of this on a surgical cutting task where the robot has to cut along a marked line in gauze.
We demonstrate cutting trajectories on a simplified cutting problem where the demonstrator quasi-statically traces the line without cutting it.
Here there are known dynamics and we can learn transitions and local rewards (e.g., learning to follow the line).
This reward can transfer to the policy learning phase, in which we can try to cut the gauze introducing unknown dynamics.
\fi


\begin{phase}[t]
\small
\DontPrintSemicolon
\caption{Reward Learning \label{alg:tsh2}}
\KwData{Demonstration $\mathcal{D}$ and sub-goals $[\rho_1,...,\rho_k]$}

Based on the transition states, segment each demonstration $d_i$ into $k$ sub-sequences where the $j^{th}$ is denoted by $d_i[j]$.

If dynamics model is available, apply MaxEnt-IRL to each set of sub-sequences $1...k$.

If the dynamics model is not available compute Equation \ref{localq} for each set of subsequences.

\KwResult{$\mathbf{R}_{seq}$}
\end{phase}

\vspace{-15pt}
\subsubsection{Model-free: Local Quadratic Rewards}
When the linearity assumption cannot be made, \hirl can alternatively use a model-free approach for reward construction.
The role of the reward function is to guide the robot to the next transition region $\rho_i$.
A straight forward thing approach is for each segment $i$, we can define a reward function as follows:
\[
R_i(x) = -\|x - \mu_{i}\|_2^2, 
\]
which is just the Euclidean distance to the centroid.

A problem with using Euclidean distance directly is that it uniformly penalizes disagreement with $\mu$ in all dimensions.
During different stages of a task, some features will likely naturally vary more than others--this is learned through IRL.
To account for this, we derive a reasonable $Q$ that is independent of the dynamics:
\[
Q[j,l] = \Sigma^{-1}_x,
\]
which is the inverse of the covariance matrix of all of the state vectors in the segment:
\begin{equation}
Q[j,l] = (\sum_{t=start}^{end} x x^T)^{-1},
\label{localq}
\end{equation}
which is a $p \times p$ matrix defined as the covariance of all of the states in the segment $i-1$ to $i$.
Intuitively, if a feature has low variance during this segment, deviation in that feature from the desired target it gets penalized. 
This is exactly the Mahalonabis distance to the next transition. 

For example, suppose one of the features $j$ measures the distance to a reference trajectory $u_t$. 
Further, suppose in step one of the task the demonstrator's actions are perfectly correlated with the trajectory ($Q_{i}[j,j]$ is low where variance is in the distance) and in step two the actions are uncorrelated with the reference trajectory ($Q_{i}[j,j]$ is high).
Thus, $Q$ will respectively penalize deviation from $\mu_{i}[j]$ more in step one than in step two.


\subsection*{Phase III. Policy Learning}
\seclabel{policy-learning}

In Phase III, \hirl uses the learned transitions $[\rho_1,...,\rho_k]$ and $\mathbf{R}_{seq}$ as rewards for a Reinforcement Learning algorithm.
In this section, we describe learning a policy $\pi$ given rewards $\mathbf{R}_{seq}$ and an ordered sequence of transitions $G$.

However, this problem is not trivial since solving $k$ independent problems neglects potential shared value structure between the local problems (e.g., a common failure state).
Furthermore, simply taking the aggregate of the rewards can lead to inconsistencies since there is nothing enforcing the order of operations.
The key insight is that a single policy can be learned jointly over all segments over a modified problem where the state-space with additional variables that keep track of the previously achieved segments.
To do so, we require an MDP model that also captures the history of the process.


\begin{phase}[t]
\small
\DontPrintSemicolon
\caption{Policy Learning \label{alg:tsh3}}
\KwData{Transition States $G$, Reward Sequence $\mathbf{R}_{seq}$, exploration parameter $\epsilon$}

Initialize $Q(\binom{s}{v},a)$ randomly

\ForEach{$iter \in 0,...,I$}{
    Draw $s_0$ from initial conditions
    
    Initialize $v$ to be $[0,...,0]$
    
    Initialize $j$ to be $1$
    
    \ForEach{$t \in 0,...,T$}{
        Choose best action $a$ based on $Q$ or random action w.p $\epsilon$.
        
        Observe Reward $R_{j}$
        
        Update state to $s'$ and $Q$ via Q-Learning update
        
        If $s'$ is $\in  \rho_{j}$ update $v[j] = 1$ and $j = j +1$
    }
}

\KwResult{Policy $\pi$}
\end{phase}

\vspace{-15pt}
\subsubsection{MDPs with Memory}
RL algorithms apply to problems that are specified as MDPs.
The challenge is that some sequential tasks may not be MDPs.
For example, attaining a reward at $\rho_i$ depends on knowing that the reward at goal $\rho_{i-1}$ was attained.
In general, to model this dependence on the past requires MDPs whose state-space also includes history.

Given a finite-horizon MDP $\mathcal{M}$ as defined in Section \ref{sec:back}, we can define an MDP $\mathcal{M}_H$ as follows.
Let $\mathcal{H}$ denote set of all dynamically feasible sequences of length smaller than $T$ comprised of the elements of $S$.
Therefore, for an agent at any time $t$, there is a sequence of previously visited states $H_t \in \mathcal{H}$.
The MDP $\mathcal{M}_H$ is defined as:
\[
\mathcal{M}_H = \langle S \times \mathcal{H},A,P'(\cdot,\cdot), R(\cdot,\cdot),T \rangle.
\]
For this MDP, $P'$ not only defines the transitions from the current state $s \mapsto s'$, but also increments the history sequence $H_{t+1} = H_{t} \sqcup s$.
Accordingly, the parametrized reward function $R$ is defined over $S$, $A$, and $H_{t+1}$.

$\mathcal{M}_H$ allows us to address the sequentiality problem since the reward is a function of the state and the history sequence.
However, without some parametrization of $H_t$, directly solving this MDPs with RL is impractical since it adds an overhead of $\mathcal{O}(e^{T})$ states.

\vspace{-15pt}
\subsubsection{Policy Learning}
Using our sequential task definition, we know that the reward transitions ($R_{i}$ to $R_{i+1}$) only depend on an arrival at the transition state $\rho_{i}$ and not any other aspect of the history.
Therefore, we can store a vector $v$, a $k$ dimensional binary vector ($v \in \{0,1\}^k$) that indicates whether a transition state $i \in 0,...,k$ has been reached.
This vector can be efficiently incremented when the current state $s \in \rho_{i+1}$.
Then, the additional complexity of representing the reward with history over $S \times  \{0,1\}^k$ is only $\mathcal{O}(k)$ instead of exponential in the time horizon.

The result is an augmented state-space $\binom{s}{v}$ to account for previous progress.
Over this state-space, we can apply Reinforcement Learning algorithms to iteratively converge to a successful policy for a new task instance.
\hirl applies Q-Learning with a Radial Basis Function value function representation to learn a policy $\pi$ over this state-space and the reward sequence $\mathbf{R}_{seq}$.
This is summarized in Algorithm~\ref{alg:tsh3}.

\section{\hirl: Policy Learning}
\seclabel{policy-learning}
If the demonstration dynamics are consistent with the execution setting, while learning the reward with MaxEnt-IRL, an optimal policy can simultaneously be extracted. However, we observed a number of practical challenges: (1) the reward function can often be represented more concisely than the policy, and as such, the demonstration data might be sufficient to estimate an accurate reward but not a useful policy, (2) the dynamics of the demonstration environment often differ slightly from the dynamics of the execution environment--making the rewards transferable but not the policies, and (3) transfer problems where the task instance has changed. 

Our solution is to refine the learned policy with physical rollouts. \hirl uses the learned transitions $[\rho_1,...,\rho_k]$ and $\mathbf{R}_{seq}$ as rewards for a Reinforcement Learning algorithm. In this section, we describe learning a policy $\pi$ given rewards $\mathbf{R}_{seq}$ and an ordered sequence of transitions $G$.
However, this problem is not trivial since solving $k$ independent problems neglects potential shared value structure between the local problems (e.g., a common failure state).
Furthermore, simply taking the aggregate of the rewards can lead to inconsistencies since there is nothing enforcing the order of operations.
The key insight is that a single policy can be learned jointly over all segments over a modified problem where the state-space with additional variables that keep track of the previously achieved segments.

\subsection{Off Policy RL Algorithms}
There are two classes of RL algorithms, on-policy algorithms (e.g., Policy Gradients, Trust Region Policy Optimization) and off-policy algirithms (e.g., Q-Learning). An on-policy algorithm learns the value of the policy being carried out by the agent and incrementally optimizes this policy. On policy are often more efficient since the robot learns to optimize the reward function in states that it is likely to visit, however, it requires that exploration is done with a specific policy that is continuously updated.
On the other hand, off-policy algorithms learn value of the optimal policy regardless of the policy used to collect the data, as long the robot sufficiently explores the space.
This is highly beneficial for our problem setting.
A single fixed exploration policy can be used to collect a large batch of data up front, which we can use to refine our model.
This is the motivation for using a Q-Learning approach in \hirl.

\subsection{Segmentation Introduces Memory}
In our sequential task definition, we cannot transition to reward $R_{i+1}$ unless all previous transition regions $\rho_{1},...\rho_{i}$ are reached in sequence.
This introduces a dependence on the history which violates the MDP structure.

Naively addressing this problem can lead to an exponential cost in terms of state-representation.
Given a finite-horizon MDP $\mathcal{M}$ as defined in Section \ref{sec:back}, we can define an MDP $\mathcal{M}_H$ as follows.
Let $\mathcal{H}$ denote set of all dynamically feasible sequences of length smaller than $T$ comprised of the elements of $S$.
Therefore, for an agent at any time $t$, there is a sequence of previously visited states $H_t \in \mathcal{H}$.
The MDP $\mathcal{M}_H$ is defined as:
\[
\mathcal{M}_H = \langle S \times \mathcal{H},A,P'(\cdot,\cdot), R(\cdot,\cdot),T \rangle.
\]
For this MDP, $P'$ not only defines the transitions from the current state $s \mapsto s'$, but also increments the history sequence $H_{t+1} = H_{t} \sqcup s$.
Accordingly, the parametrized reward function $R$ is defined over $S$, $A$, and $H_{t+1}$.
$\mathcal{M}_H$ allows us to address the sequentiality problem since the reward is a function of the state and the history sequence.
However, without some parametrization of $H_t$, directly solving this MDPs with RL is impractical since it adds an overhead of $\mathcal{O}(e^{T})$ states.

Our key insight is to the leverage the definition of the Markov Segmentation function formalized earlier.
We know that the reward transitions ($R_{i}$ to $R_{i+1}$) only depend on an arrival at the transition state $\rho_{i}$ and not any other aspect of the history.
Therefore, we can store a vector $v$, a $k$ dimensional binary vector ($v \in \{0,1\}^k$) that indicates whether a transition state $i \in 0,...,k$ has been reached.
This vector can be efficiently incremented when the current state $s \in \rho_{i+1}$.
The result is an augmented state-space $\binom{s}{v}$ to account for previous progress.
Then, the additional complexity of representing the reward with history over $S \times  \{0,1\}^k$ is only $\mathcal{O}(k)$ instead of exponential in the time horizon.

\subsubsection{Policy Learning}
Over this state-space, we can apply Reinforcement Learning algorithms to iteratively converge to a successful policy for a new task instance.
\hirl applies Q-Learning with a Radial Basis Function value function representation to learn a policy $\pi$ over this state-space and the reward sequence $\mathbf{R}_{seq}$.
This is summarized in Algorithm~\ref{alg:tsh3}.



\begin{phase}[t]
\small
\DontPrintSemicolon
\caption{Policy Learning \label{alg:tsh3}}
\KwData{Transition States $G$, Reward Sequence $\mathbf{R}_{seq}$, exploration parameter $\epsilon$}

Initialize $Q(\binom{s}{v},a)$ randomly

\ForEach{$iter \in 0,...,I$}{
    Draw $s_0$ from initial conditions
    
    Initialize $v$ to be $[0,...,0]$
    
    Initialize $j$ to be $1$
    
    \ForEach{$t \in 0,...,T$}{
        Choose best action $a$ based on $Q$ or random action w.p $\epsilon$.
        
        Observe Reward $R_{j}$
        
        Update state to $s'$ and $Q$ via Q-Learning update
        
        If $s'$ is $\in  \rho_{j}$ update $v[j] = 1$ and $j = j +1$
    }
}

\KwResult{Policy $\pi$}
\end{phase}








%\documentclass[0-main.tex]{subfiles}
%\begin{document}

\section{Experiments}\label{sec:exp}
We evaluate \hirl on two standard RL benchmarks and in deformable cutting and tensioning on the da Vinci surgical robot.

\begin{figure}[t]
\centering
 \includegraphics[width=\columnwidth]{figures/domains.png}
 \caption{(A) Simulated control task with a car with noisy non-holonomic dynamics. The car ($A_1$) is controlled by accelerating and turning in discrete increments. The task is to park the car between two obstacles. \label{domains}}
\end{figure}

\begin{figure}[t]
\centering
 \includegraphics[width=\columnwidth]{exp/rc-car-segmentation.png}
 \caption{(Left) the 5 demonstration trajectories for the parallel parking task, and (Right) the sub-goals learned by \hirl. There are two intermediate goals corresponding to positioning the car and orienting the car correctly before reversing. \label{exp:rcsegmentation}}
\end{figure}

\subsection{Fully Observed Parallel Parking}\label{exp:pp}
We constructed a parallel parking scenario for a robot car with non-holonomic dynamics and two obstacles (Figure \ref{domains}a). 
The car can accelerate or decelerate in discrete $\pm 0.1$ meters per second increments (and reverse), and change its heading by $5^\circ$ degree increments.
The car's speed ($\|\dot{x}\|+\|\dot{y}\|$) and heading ($\theta$) are inputs to a bicycle steering model which computes the next state.
The car observe its x position, y position, orientation, and speed in a global coordinate frame.
The robot's dynamics are noisy and with probability 0.1 will randomly add or subtract $2.5^\circ$ degrees to the steering angle.
If the robot parks between the obstacles, i.e., 0 speed within a $15^\circ$ tolerance and a positional tolerance of $5$ meters, the task is a success and the robot receives a reward of $1$. 
If the robot collides with one of the obstacle or does not park in 200 timesteps the episode ends with a reward of $0$.

We call this domain Parallel Parking with Full Observation (PP-FO). We consider the following approaches:

\vspace{0.25em}\noindent \textbf{RL (Q-Learning): } The baseline approach is modeling the entire problem as an MDP with the sparse delayed reward. We apply Q-Learning to learn a policy for this problem with a radial basis function representation for the Q function with number of bases and bandwidth $k=5, \sigma=0.1$ respectively. The radial basis function hyper-parameters were tuned manually to achieve the fastest convergence in the experimental task. 

\vspace{0.25em}\noindent \textbf{Behavioral Cloning (SVM): } We generated $N$ demonstrations using an RRT motion planner (assuming deterministic dynamics). The next baseline is to directly learn a policy from the generated plans using behavioral cloning. We use a L1 hinge-loss SVM with L2 regularization $\alpha=5e-3$ to predict the action from the state. The hyper-parameters were tuned manually using cross-validation by holding out trajectories.

\vspace{0.25em}\noindent \textbf{Single-Step IRL (MaxEnt-IRL): } We generated $N$ demonstrations using an RRT motion planner (assuming deterministic dynamics). We use the collected demonstrations and infer a quadratic reward function using MaxEnt-IRL (both using estimated dynamics and ground truth dynamics). The learned reward function is optimized using Q-learning with a radial basis function representation with the same hyper-parameters as the RL approach. 

\vspace{0.25em}\noindent \textbf{\hirl: } Finally, we apply \hirl to the $N$ demonstrations, learn segmentation, and quadratic rewards (Figure~\ref{exp:rcsegmentation}).
We apply \hirl with a DP-GMM based segmentation step with no kernel transformation (as described in Section \ref{segm}).
For the local IRL approach, we consider three approaches: MaxEnt with ground truth dynamics, MaxEnt with locally estimated dynamics, Model-Free. 
The learned reward functions and transition regions are used in the policy learning phase with Q-learning with a radial basis function representation with the same hyper-parameters as the RL approach.

\vspace{0.5em}

\subsubsection{Fixed Demonstrations, Vary Rollouts}
In the first experiment, we fix number of initial demonstrations $N=5$, and vary the number of rollouts.


\subsubsection{Fixed Rollouts, Vary Demonstrations}
Next, we fix the number of rollouts to $1250$, and vary the number of demonstration trajectories each approach observes.
In the fully observed problem, compared to MaxEnt-IRL, the model-based \hirl converges to a policy with a 60\% success rate with about 3x fewer time-steps.
The gains for the model-free version are more modest with a 50\% reduction.
The supervised policy learning approach achieves a success rate of 47\% and the baseline RL approach achieves a success rate of 36\% after 250000 time-steps.

The baseline Q-Learning approach directly tries to learn a sequence of actions to minimize the quadratic cost around the target state. 
This leads to a lot of exploration since the robot must first make ``negative'' progress (pulling forward). 
\hirl improves convergence since it structures the exploration through the segmentation.
The local reward functions are better shaped to guide the car towards its short term goal.
This focuses the exploration on solving the short term problem first.
MaxEnt-IRL mitigates some of the problems since it rewards states based on their estimated cost-to-go, but as the time-horizon increases the estimates of this become nosier--leading to worse performance (see technical report for a characterization~\cite{krishnan2016hirl}).


\subsubsection{Vary Task Parameters}
Finally, we explore how well the constructed rewards transfer if the dynamics are perturbed in the fully observed setting.
We expect MaxEnt-IRL to transfer well because it learns a delayed reward, which tends to encode success conditions and not task-specific details.
After constructing the rewards, we randomly perturbed the system dynamics by introducing a bias towards turning left.
We find that the model-based \hirl technique transfers to this domain comparably to MaxEnt-IRL until the task is so different that the sub-goals learned with \hirl are no longer informative.
The model-free \hirl algorithm converges more slowly; requiring 20\% more time-steps to converge to the same success rate. 



\subsection{Partially Observed Parallel Parking}
Next, we made the  Parallel Parking domain a little harder. We hid the velocity state from the robot, so the robot only sees $(x,y,\theta)$. As before, if the robot collides with one of the obstacle or does not park in 200 timesteps the episode ends.
We call this domain Parallel Parking with Partial Observation (PP-PO).

As before, we generated 5 demonstrations using an RRT motion planner (assuming deterministic dynamics) and applied \hirl to learn the segments.
Figure \ref{exp:rcsegmentation} illustrates the demonstrations and the learned segments. 


In the partial observation problem (PP-PO), there is no longer a stationary policy that can achieve the reward.
The techniques that model this problem with a single MDP all fail to converge.
The learned segments in \hirl help disambiguate dependence on history.
After 250000 time-steps, the policy learned with model-based \hirl has a 70\% success rate in comparison to a <10\% success rate for the baseline RL, MaxEnt-IRL, and 0\% for the SVM.









\begin{figure}[t]
\centering
 \includegraphics[width=0.32\columnwidth]{concept/swirl1-rewards.png}
  \includegraphics[width=0.32\columnwidth]{concept/swirl1-linear.png}
   \includegraphics[width=0.32\columnwidth]{concept/swirl1-quadratic.png}
 \caption{This plot illustrates a conceptual GridWorld task. The green square denotes the starting position and the red denotes the target. The first plot shows the basic task, the second plot shows the reward learned with IRL from 5 demonstrations with a linear reward parametrization, and the second plot shows the reward learned with a quadratic parametrization. \label{concept:1}}
\end{figure}

\subsection{Conceptual Example: Discrete Planning}
We start with a GridWorld example to illustrate the structure of tasks amenable to segmentation. 

\vspace{0.5em} \noindent \textbf{Motivating Experiments: } Before we discuss segmentation, the first hypothesis that we have to evaluate is whether IRL on its own can improve the convergence of RL. This hypothesis is not unreasonable since some problem have a naturally sparse reward function (e.g., 1 if a goal state is reached, 0 else where), and IRL would approximate this reward with a smoother quadratic function. So consider an 16 x 16 GridWorld with a start position in one corner and a goal in another corner (Figure \ref{concept:1}a). 
The agent can move left, right, up, and down, and with a 30\% probability the action results in a random motion.
We sampled 5 demonstrations with a value iteration supervisor and applied IRL with linear and quadratic rewards. These results are visualized in (Figure \ref{concept:1}b-c).
These plots illustrate how IRL can be used to shape a reward function, by changing the reward parametrization. Even though the underlying reward function is a sparse 1/0 reward,  MaxEnt fits a linear or a quadratic reward function, which is smoother. These smoother reward functions can help guide the agent to the goal.

\begin{figure}[t]
\centering
 \includegraphics[width=0.48\columnwidth]{concept/1.png}
  \includegraphics[width=0.48\columnwidth]{concept/2.png}
 \caption{(A) Tabular Q-Learning converges nearly 8x faster when the reward is quadratically shaped by IRL. (B) These benefits are even more pronounced when the Q function is approximated by a linear regression model. \label{concept:2}}
\end{figure}

If we were to compare the convergence of RL on the 1/0 reward and the quadratic reward, we find vastly different convergence properties.
We use a tabular Q-Learning agent with an $\epsilon$-greedy exploration policy ($\epsilon=0.1$).
Q-Learning on the quadratic reward converges in nearly 8x fewer steps than the 1/0 reward (Figure \ref{concept:2}a).
This is because the Q-Learning agent with the shaped reward function observes a reward earlier than the 0/1 agent, and thus, it is able to make progress towards the goal earlier in the learning process.
Combining function approximation with Q-Learning allows it to generalize to nearby unseen states.
The effects are even more pronounced since now once the agent observes the ``right direction'' to travel due to the shaped reward it is able to quickly make progress (Figure \ref{concept:2}b).
In the following experiments, we will use the term \textsf{IRL} to refer to the combined IRL+RL policy inference procedure.

\begin{figure}[t]
\centering
 \includegraphics[width=0.48\columnwidth]{concept/3.png}
  \includegraphics[width=0.48\columnwidth]{concept/4.png}
 \caption{(A) Given 5 expert demonstrations, Behavioral Cloning with a Random Forest classifier achieves a high reward on the same task. This can be used to initialize the Q-Learning agents. On the same task there is a marginal benefit to IRL. (B) We perturb the task after collecting 5 demonstrations by adding an obstacle blocking the shortest path, and find that the IRL agent is the most effective.  \label{concept:3}}
\end{figure}

One could hand-craft such reward functions, but one advantage of IRL is that it learns such functions from demonstration data. Admittedly, the previous comparison with RL is not completely fair. The RL algorithm with the 1/0 reward does not use the demonstration data. In principle, one could train a policy with Behavioral Cloning and use that to initialize the RL agent. Given the same 5 initial demonstrations, we train a random forest classifier.
When we apply this policy to the same task instance where the demonstrations were collected, the behavioral cloning policy does very well (Figure \ref{concept:3}a).
This can be used as an initialization to the Q-learning agent, which can further improve the policy.
On the same task instance, there is a marginal benefit to using IRL to shape the reward--as the behavioral cloning policy already gets the agent very close to the goal in most cases.

However, suppose that demonstration domain slightly differs from the execution domain. We simulate this by adding a 4x4 obstacle in the center of the GridWorld map.
Now, the learned behavioral cloning policy is not effective on the new domain (Figure \ref{concept:3}b).
However, the reward function learned with IRL transfers more robustly.
Furthermore, combining the IRL with a BC initialization improves performance over RL by 6x.
These results motivate us to consider how to use demonstrations to improve the convergence of RL.
In more complex tasks, a quadratic approximation of the reward may not suffice.
Hence, we consider how to segment the task into sub-tasks that can be approximated with a quadratic reward.


\begin{figure}[t]
\centering
 \includegraphics[width=\columnwidth]{concept/swirl-rewards.png}
 \caption{(A) A GridWorld domain with an obstacle, (B) Visualization of the reward function if we apply IRL to 5 demonstrations and fit a quadratic reward. The quadratic reward can encourage the agent to get ``stuck'' in the obstacle if it is too greedy, (C) Segmented quadratic reward function learned with \hirl. The function has two components first guiding the agent to the passage on the left, and then guiding to the goal.  \label{concept:4}}
\end{figure}

\begin{figure}[t]
\centering
 \includegraphics[width=0.6\columnwidth]{concept/2-1.png}
 \caption{\hirl converges faster than a single quadratic reward or the 1-0 reward. \label{concept:5}}
\end{figure}


\vspace{0.5em} \noindent \textbf{Hypothesis 0. IRL Can Benefit From Segmentation: } Now, we make the domain above slightly more complicated. We add an obstacle in such a way that the straight-line path is no longer optimal (Figure \ref{concept:4}a). As before, we sample 5 demonstrations from a value iteration supervisor.
Qualitatively, the quadratic reward learned from IRL is misleading as it can guide an overly greedy agent into the obstacle(Figure \ref{concept:4}b). 
\hirl learns a two-segment reward, where first it guides the agent to a point in the left passage and then a reward around the goal (Figure \ref{concept:4}c).
The number of segments was determined by a Dirichlet process prior as described in the text.

To evaluate \hirl quantitatively, we use a tabular Q-learning agent to learn a policy using these rewards.
To control for initialization effects, the agents were initialized randomly (unlike the previous experiment) and used an $\epsilon=0.1$ exploration policy.
This results in a significant improvement in convergence as seen in Figure \label{concept:5}.

\begin{figure}[t]
\centering
 \includegraphics[width=0.48\columnwidth]{concept/3-1.png}
  \includegraphics[width=0.48\columnwidth]{concept/3-2.png}
 \caption{(A) We measure the sensitivity to the number of initial demonstrations, (B) We perturb the execution environment by adding random obstacles. \label{concept:6}}
\end{figure}

\vspace{0.5em} \noindent \textbf{Parameter Sensitivity: } Finally, we use the previous GridWorld environment to illustrate the sensitivity to different parameters. In Figure \ref{concept:6}a, we vary the number of initial demonstrations provided to IRL and Behavioral Cloning and measure the performance after 1500 rollouts.
IRL is not as sensitive to the number of demonstrations as BC.
This is because the reward function that IRL is estimating is much simpler than policy function.
Next, we evaluate each of the algorithms on their ability to generalize to different task instances.
We perturb the execution environment by randomly adding single grid point obstacles.
We average the results over 50 such random perturbations (Figure \ref{concept:6}b).

Not surprisingly, we find that IRL is the most robust.
\hirl is relatively robust but is slightly worse than IRL for a large number of random obstacles.
We wanted to understand why \hirl was less robust than standard IRL, so we manually set the number of segments in \hirl to $4$.
The performance of \hirl with 4 segments is much worse.
This is because obstacles can ``invalidate'' segments.
This suggests an interesting tradeoff where more segments can serve to more precisely guide the agent towards the goal, but less segments lead to improved generalization.

\subsection{Simulated Control Domains}
Next, we describe \hirl on simulated control domains. These illustrate examples with continuous state-spaces.







\begin{figure*}[t]
\centering
 \includegraphics[width=0.8\textwidth]{exp/rc-convergence-1.png}
 \caption{Performance on a parallel parking task with noisy dynamics with full state observations (position, orientation, and velocity), partial observation (only position and orientation), and transfer (randomly permuting the action space).
 Success is measured in terms of the probability that the car successfully parked, and (M) denotes whether the approach used the dynamics model.
 In the fully observed case, both the model-based and model-free \hirl algorithms converge faster than MaxEnt-IRL and quickly outperforms the SVM.
 In the partially observed case, MaxEnt-IRL, Q-Learning, and the SVM fail--while \hirl succeeds.
 Both techniques also demonstrate comparable transferability to MaxEnt-IRL when the domain's dynamics are perturbed.
\label{exp:rcsegmentation-res}}
\end{figure*}



\begin{figure*}[t]
\centering
 \includegraphics[width=0.8\textwidth]{exp/acr-convergence-1.png}
 \caption{Acrobot: We measured the performance of rewards constructed with \hirl and the alternatives. We find that \hirl (model-based and model-free) converges faster than MaxEnt-IRL, Q-Learning, and the SVM.
 Furthermore, \hirl requires less demonstrations, which we measure by comparing the performance of the alternatives after a fixed 50000 time-steps and with varied input demonstrations. 
 We also vary the task parameters by changing the size of the second link of the pendulum and find that the learned rewards are robust to this variation (as before comparing the performance of the alternatives after a fixed 50000 time steps). MaxEnt-IRL shows improved transfer performance since once the task has changed enough the segments learned during the demonstrations may not be informative and may even hurt performance if they are misleading. 
 \label{exp:acsegmentation-res2}}
\end{figure*}

\begin{figure}[t]
\centering
    \includegraphics[width=0.8\columnwidth]{exp/IMAG0249.jpg}
    \caption{The experimental setup for gauze grasping and tensioning. The task is to grasp the gauze and lift it to flatten it out. This task is not usually successful in an open-loop trajectory due to deformation after the grasp.
    }
    \label{exp:dvrk2}
% \vspace{-15pt}
\end{figure}

\begin{figure}[t]

    \includegraphics[width=\columnwidth]{exp/signals.png}
    \raggedright
    \includegraphics[width=0.9\columnwidth]{exp/segmentation.png}
    \caption{A representative demonstration of the deformable sheet grasping task with relevant features plotted over time. \hirl identifies 4 segments when applied to the deformable sheet grasping task. 
    }
    \label{exp:dvrk3}
% \vspace{-15pt}
\end{figure}

\subsubsection{Acrobot}\label{exp:acrobot}
This domain consists of a two-link pendulum with gravity and with torque controls on the joint. The dynamics are noisy and there are limits on the applied torque. The robot has 1000 timesteps to raise the arm above horizontal ($y=1$ in the images). If the task is successful and the robot receives a reward of $1$. 
Thus, the expected reward is equivalent to the probability that the current policy will successfully raise the arm above horizontal.
We generated $N=5$ demonstrations for the Acrobot task and applied segmentation. 
These demonstrations were generated by training the Q-Learning baseline to convergence and then sampling from the learned policy.
%We visualize the learned segments in Figure \ref{exp:acroseg}, which can be seen to qualitatively describe a successful path.
In Figure \ref{exp:acsegmentation-res2}, we plot the performance of the all of the approaches.
We include a comparison between a Linear Multiclass SVM and a Kernelized Multiclass SVM for the policy learning alternative.
In this example, we find that applying MaxEnt-IRL does not improve the convergence rate.
For this state-space, MaxEnt-IRL merely recovers the reward used in the original RL problem.
On the other hand, added segments using \hirl improve convergence rates.

% the \hirl approaches add segments making it easier to converge.

We also vary the number of input demonstrations to \hirl and find that it requires fewer demonstrations than policy learning and MaxEnt-IRL to converge to a more reliable policy.
It takes about 10x more demonstrations for the supervised learning approach to reach comparable reliability.
Finally, we find that \hirl does not sacrifice much transferability.
We learn the rewards on the standard pendulum, and then during learning we vary the size of the second link in the pendulum.
We plot the success rate (after a fixed 50000 steps) as a function of the increase link size.
\hirl is significantly more robust than supervised policy learning to the increase in link size and has a significantly higher success rate than IRL for small perturbations in link size. 

\subsection{Physical Experiments with the da Vinci Surgical Robot}


\subsubsection{Deformable Sheet Grasping and Tensioning: } Next, we apply \hirl to learn to how to grasp and tension a sheet of gauze. The experimental setup is pictured in Figure \ref{exp:dvrk2}. The basic setup is a sheet of gauze fixtured at the two far corners. The robot's task to the grasp the gauze on the opposing side, lift it up, and in the process flatten the sheet out.
This task is not usually successful in an open-loop trajectory due to deformation after the grasp.
Some grasps pick up more or less of the material and the flattening procedure has to be accordingly modified.
The state-space is the 6 DoF end-effector position of the robot, the current load on the wrist of the robot, and a visual feature measuring the differences in disparity markers on the sheet (a proxy for flatness).
The action space is discretized into an 8 dimensional vector ($\pm x$, $\pm y$, $\pm z$, open/close gripper) where the robot moves in 2mm increments.

We provided 15 demonstrations through a keyboard based tele-operation interface.
From these 15 demonstrations, \hirl identifies four segments. Figure \ref{exp:dvrk3} illustrates the segmentation of a representative demonstration with important states plotted over time.
In this problem, we applied both the model-based and model-free versions of \hirl to construct the rewards.
One of the segments corresponds to moving to the correct grasping position, one corresponds to making the grasp, one lifting the gauze up again, and one corresponds to straightening the gauze.

We can use the reward functions learned by \hirl to refine the policy for this task.
We define a Q-Network with a single-layer Multi-Layer Perceptron with 32 hidden units.
For each of the segments, we apply Behavioral Cloning locally with the same architecture as the Q-network (with an additional softmax over the output layer) to get an initial policy. We rollout 100 trials with an $\epsilon=0.1$ greedy version of these segmented policies.
The learning results of this experiment are summarized below with different variations of the learning algorithms.
The value of the policy is a measure of average disparity over the gauze accumulated over the task (if the gauze is flatter longer, then the value is greater).
As a baseline, we applied RL for 100 rollouts with no other information. RL did not successfully grasp the gauze even once.
Next, we applied behavioral cloning (BC) directly.
BC was only able to reach the gauze and distrub it but not succesfully grasping.
Then, we applied the segmentation from \hirl  and applied BC directly to each local segment (without further refinement). 
This was able complete the full task.
Finally, we applied all of \hirl and found the highest-value results.
For comparison, we applied \hirl without the BC initialization and found that it was only successful at the first two steps.

% Please add the following required packages to your document preamble:
% \usepackage[table,xcdraw]{xcolor}
% If you use beamer only pass "xcolor=table" option, i.e. \documentclass[xcolor=table]{beamer}
\begin{table}[]
\centering
\scriptsize
\caption{Results from the deformable sheet grasping experiment}
\label{my-label}
\begin{tabular}{llll}
\rowcolor[HTML]{000000} 
{\color[HTML]{FFFFFF} Technique} & {\color[HTML]{FFFFFF} \# Demonstrations} & {\color[HTML]{FFFFFF} \# Rollouts} & {\color[HTML]{FFFFFF} Value} \\
RL (ab initio)                   & -                                        & 100                                & -8210                        \\
BC                               & 15                                       & -                                  & -7591                        \\
Segmentation+BC                  & 15                                       & -                                  & -3516                        \\
SWIRL+MF (no init)                  & 15                                       & 100                                & -6128                        \\
SWIRL+MB (no init)                  & 15                                       & 100                                & -5798                        \\
\textbf{SWIRL+MF (BC init)}                  & 15                                       & 100                                & \textbf{-3110}     \\
\textbf{SWIRL+MB (BC init)}                  & 15                                       & 100                                & \textbf{-2241}                       
\end{tabular}
\end{table}

\subsubsection{Surgical Line Cutting: }
In the next experiment, we evaluate generalization to different task instances.
We apply \hirl to learn to cut along a marked line in gauze similar to Murali et al.~\cite{murali2015learning}.
This is a multi-step problem where the robot starts from a random initial state, has to move to a position that allows it to start the cut, and then cut along the marked line.
We provide the robot 5 kinesthetic demonstrations by positioning the end-effector and then following various marked straight lines.
The state-space of the robot included the end-effector position $(x,y)$ as well as a visual feature indicating its pixel distance to the marked line $(pix)$.
This visual feature is constructed using OpenCV thresholding for the black line.
Since the gauze is planar, the robot's actions are unit steps in the $\pm x, \pm y$ axes.
Figure\,\ref{exp:dvrk1} illustrates the training and test scenarios.

\hirl identifies two segments corresponding to the positioning step and the termination.
The learned reward function for the position step minimizes $x,y,pix$ distance to the starting point and for the cutting step the reward function is more heavily weighted to minimize the $pix$ distance.
We defined task success as positioning within $1$\,cm of the starting position of the line and during the following stage, missing the line by no more than $1$\,cm (estimated from pixel distance).
We evaluated the model-free version of \hirl, Q-Learning, and Behavioral Cloning with an SVM.
\hirl was the only technique able to achieve the combined task.
This is because the policy for this task is non-stationary, and \hirl is the only approach of the alternatives that can learn such a policy.

We evaluated the learned tracking policy to cut gauze.
We ran trials on different sequences of curves and straight lines. 
Out of the 15 trials, 11 were successful.
2 failed due to \hirl errors (tracking or position was imprecise) and 2 failed due to cutting errors (gauze deformed causing the task to fail).
1 of the failures was on the 4.5 cm curvature line and 3 were one the 3.5 cm curvature line.

% \begin{figure}[t]
% \centering
%  \includegraphics[width=0.7\textwidth]{exp/dvrk-demos-1.png}
%  \caption{We collected demonstrations on the da Vinci surgical robot kinesthetically. The task was to cut a marked line on gauze. We demonstrated the location of the line without actually cutting it. The goal is to infer that the demonstrator's reward function has two steps: position at a start position before the line, and then following the line. We applied this same reward to lines that were not straight nor started in exactly the same position.\label{exp:dvrk1}}
% \end{figure}

% \begin{SCfigure}[10][t]
%     \centering
%     \vspace{-0.5em}
%     \includegraphics[width=0.5\textwidth]{exp/dvrk-demos-1.png}
%     \caption{We collected demonstrations on the da Vinci surgical robot kinesthetically. The task was to cut a marked line on gauze. We demonstrated the location of the line without actually cutting it. The goal is to infer that the demonstrator's reward function has two steps: position at a start position before the line, and then following the line. We applied this same reward to lines that were not straight nor started in exactly the same position.}
%     \label{exp:dvrk1}
%     \vspace{-1.5em}
% \end{SCfigure}


\begin{figure}[t]
\centering
    \includegraphics[width=\columnwidth]{exp/dvrk-demo-2.png}
    \caption{
      We collected demonstrations on the da Vinci surgical robot kinesthetically. The task was to cut a marked line on gauze. We demonstrated the location of the line without actually cutting it. The goal is to infer that the demonstrator's reward function has two steps: position at a start position before the line, and then following the line. We applied this same reward to curved lines that started in different positions.
    }
    \label{exp:dvrk1}
% \vspace{-15pt}
\end{figure}

\begin{table*}[ht]
    \centering
    \caption{With 5 kinesthetic demonstrations of following marked straight lines on gauze, we applied \hirl to learn to follow lines of various curvature. After 25 episodes of exploration, we evaluated the policies on the ability to position in the correct cutting location and track the line. We compare to SVM on each individual segment. SVM is comparably accurate on the straight line (training set) but does not generalize well to the curved lines.
    \label{dvrk:res1}}
    \resizebox{\linewidth}{!}{% put in textwidth
    \begin{tabular}{c||c|c|c|c}
    \hline
    \rowcolor[HTML]{CBCEFB} 
    Curvature Radius (cm) & SVM Pos. Error (cm) & SVM Tracking Error (cm) & \hirl Pos. Error (cm) & \hirl Tracking Error (cm) \\
     \hline \hline
    straight & 0.46 & 0.23 & 0.42 & 0.21  \\
    \rowcolor[HTML]{E0E0E0} 
    4.0 & 0.43 & 0.59 & 0.45 & 0.33 \\
    3.5 & 0.51 & {\color{red}\textbf{1.21}} & 0.56 & 0.38 \\
    \rowcolor[HTML]{E0E0E0} 
    3.0 & 0.86 & {\color{red}\textbf{3.03}} & 0.66 & 0.57 \\
    2.5 & {\color{red}\textbf{1.43}} & {\color{red}-} & 0.74 & 0.87 \\
    \rowcolor[HTML]{E0E0E0} 
    2.0 & {\color{red}}- & {\color{red}}- & 0.87 & {\color{red}\textbf{1.45}} \\
    1.5 & {\color{red}}- & {\color{red}}- & {\color{red}\textbf{1.12}} & {\color{red}\textbf{2.44}} \\
     \hline
    \end{tabular}
    }
    % \vspace{-10pt}
\end{table*}

Next, we characterized the repeatability of the learned policy.
We applied \hirl to lines of various curvature spanning from straight lines to a curvature radius of 1.5 cm.
Table \ref{dvrk:res1} summarizes the results on lines of various curvature.
While the SVM approach did not work on the combined task, we evaluated its accuracy on each individual step to illustrate the benefits of \hirl.
On following straight lines, SVM was comparable to \hirl in terms of accuracy.
However, as the lines become increasingly curved, \hirl generalizes more robustly than the SVM.



\section{Discussion and Future Work}
\textbf{SK. TODO revise}
\hirl is a three-phase algorithm that first segments a task, learns local rewards, and then learns a policy.
Experimental results suggest that sequential segmentation can indeed improve convergence in RL problems with delayed rewards.
Results suggest that \hirl is robust to perturbations in initial conditions, the environment, and sensing noise.
This paper formalizes the interaction and composability of the three phases (sequence, reward, and policy learning). In future work, we will explore extensions to each of the phases and quantify the degree of generalization.
We will explore how the Q-Learning step could be replaced with Guided Policy Search, Policy Gradients, and optimal control.
We will modify the segmentation algorithm to incorporate more complex transition conditions and allow for sub-optimal demonstrations.
We will explore more robotic tasks including suturing, surgical knot tying, and assembly.
Another avenue for future work is modeling complex tasks as hierarchies of MDPs.

% Another avenue for future work is modeling complex tasks as hierarchies of MDPs, namely, tasks composed of multiple MDPs that switch upon certain states and the switching dynamics can be modeled as another MDP. 

\vspace{0.5em}

{\footnotesize 
\noindent \textbf{Acknowledgements:}
This research was performed at the AUTOLAB at UC Berkeley in
affiliation with the AMP Lab, BAIR, and the CITRIS "People and Robots" (CPAR) Initiative in affiliation with UC Berkeley's Center for Automation and Learning for Medical Robotics (Cal-MR). The authors were supported in part by the U.S. National Science Foundation under NRI Award IIS-1227536: Multilateral Manipulation by Human-Robot Collaborative Systems, and by Google, UC Berkeley's Algorithms, Machines, and People Lab, Knut \& Alice Wallenberg Foundation, and by a major equipment grant from Intuitive Surgical and by generous donations from Andy Chou and Susan and Deepak Lim. We thank our colleagues and the anonymous WAFR reviewers who provided valuable feedback and suggestions, in particular, Pieter Abbeel, Anca Dragan, and Roy Fox.}


%\href{http://j.mp/v-tsc}{j.mp/v-tsc}

\bibliographystyle{SageV}
\bibliography{deepP2P,gmm,gmm2,gmm3}

%\appendix
%\section{Appendix}

\subsection{Sequence of Stable Feedback Controllers}\label{sec:appendix1}
The proposed model naturally arises from a system controlled with linear state feedback controllers to the centroids of the $k$ target regions $[\rho_1,...,\rho_k]$. 
In the Transition State model,  $[\rho_1,...,\rho_k]$ are defined as the sublevel sets of multivariate Gaussian distributions.
For each of the Gaussian mixture components, let $[\mu_1,...,\mu_k]$ denote the respective expectations and $[\Sigma_1,...,\Sigma_k]$ denote the respective covariances.
We can show that the Transition State Clustering model naturally follows from of a sequence of stable linear full-state feedback controllers sequentially controlling the system to each $\mu_i$ (up-to some tolerance defined by $\alpha$).

Suppose, we model the agent's trajectory in feature space as a linear dynamical system with a fixed dynamics.
Let $A_r$ model the agent's linear dynamics and $B_r$ model the agent's control matrix:
\[
\mathbf{x}(t+1) = A_r\mathbf{x}(t) + B_r\mathbf{u}(t) + W(t).
\]
For a particular mixture component $i$, the agent applies a linear feedback controller with gain $C_i$, regulating around the target state $\mu_i$.
This can be represented as the following system (by setting $u(t)=-C_i\hat{\mathbf{x}}$):
\[
\hat{\mathbf{x}}(t) = \mathbf{x}(t) - \mu_i.
\]
\[
\hat{\mathbf{x}}(t+1) = (A_r-B_rC_i)\hat{\mathbf{x}}(t)+ W(t).
\]
If this system is stable, it will converge to the state $\hat{\mathbf{x}}(t) = \mathbf{0}$ which is  $\mathbf{x}(t) = \mu_i$ as $t \rightarrow \infty$.
However, since this is a finite time problem, we model a stopping condition, namely, the system is close enough to $\mathbf{0}$.
For some $z_\alpha$ (e.g., in 1 dimension 95\% quantiles are $Z_{5\%} = 1.96$):
\[
\hat{\mathbf{x}}(t)^T \Sigma^{-1}_i \hat{\mathbf{x}}(t) \le z_\alpha.
\]
If the agent's trajectory was modeled as a sequence $1...K$ of such controllers, we would observe the Transition State Clustering model with each $A_i = A_r-B_rC_i$, and the clusters would be an estimate of the $(\mu_i, \Sigma_i)$.

\subsection{Mixture Models And Linear Systems}\label{sec:appendix2}
Using a GMM to detect switches in local linearity is an approximate algorithm that has been applied in a number of prior works~\cite{moldovan2013dirichlet,calinon2014task, khansari2011learning}.
This is akin to a using a Gaussian kernel for kernelized change point detection~\cite{harchaoui2009kernel}.
We provide some intuition on why this model is sensible for our application.

Consider the following dynamical system:
\[
x_{t+1} = f(x_{t}) + w_{t}
\]
where $w_{t}$ is unit-variance i.i.d Gaussian noise $N(0,I)$.
Let us first focus on linear systems.
If $f$ is linear, then the problem of learning $f$ reduces to linear regression:
\[
\arg\min_{A} \sum_{t=1}^{T-1} \|Ax_{t} - x_{t+1}\|.
\]
Alternatively, we can think about this linear regression probabilistically.
Let us first consider the following proposition:

\vspace{0.75em}

\begin{proposition}
Consider the one-step dynamics of a linear system.
Let $x_{t} \sim N(\mu, \Sigma)$, then $\binom{x_{t}}{x_{t+1}}$ is a multivariate Gaussian.
\end{proposition}
\begin{proof}
This follows from the fact that $x_{t+1}$ can be expressed as a linear combination of independent multivariate Gaussian random variables.
\end{proof}

\vspace{0.5em}

Following from this idea, if we let $p$ define a distribution over $x_{t+1}$ and $x_{t}$:
\[
p(x_{t+1},x_{t}) \sim Normal
\]
For multivariate Gaussians the conditional expectation is a linear estimate, and we can see that it is equivalent to the regression above:
\[
\arg\min_{A} \sum_{t=1}^{T-1} \|Ax_{t} - x_{t+1}\| = \mathbf{E}[x_{t+1} \mid x_{t}].
\]

The GMM model allows us to extend this line of reasoning to consider more complicated $f$.
If $f$ is non-linear $p$ will almost certainly not be Gaussian.
However, GMM models can model complex distributions in terms of Gaussian Mixture Components:
\[
p(x_{t+1},x_{t}) \sim GMM(k)
\]
where $k$ denotes the number of mixture components.
The interesting part about this mixture distribution is that locally, it models the dynamics as before.
Conditioned on particular Gaussian component $i$ the conditional expectation is:
\[
\mathbf{E}[x_{t+1} \mid x_{t}, i \in 1...k].
\]
As before, conditional expectations of Gaussian random variables are linear, with some additional weighting $\phi(i \mid x_{t},x_{t+1})$:
\[
\arg\min_{A_i} \sum_{t=1}^{T-1} \phi(i \mid x_{t},x_{t+1}) \cdot \|A_ix_{t} - x_{t+1}\|.
\]
Every tuple $(x_{t+1},x_{t})$ probability $\phi(i \mid x_{t},x_{t+1})$ of belonging to each $i$th component, and this can be thought of as a likelihood of belonging to a given locally linear model.

\subsection{Alternative Approaches}
We consider the following alternative approaches to compare against \hirl. 

\subsubsection{RL}
This approach considers no segmentation and no history. It directly applies forward RL to the apparent state-space and uses a distance-to-goal reward function.

\subsubsection{Sliding Window}
This approach considers no segmentation but includes a sliding window of $k$ previous states in the state-space. It directly applies forward RL to the augmented state-space and uses a distance-to-goal reward function.

\subsubsection{IRL}
This approach uses MaxEnt-IRL to learn a reward function without segmentation and requires $N=5$ demonstrations. We apply forward RL to the learned reward function. 

\subsubsection{Endpoint Model}
This is a simplified approach to construct rewards using the learned $G$.
Let $\{\mu_1,...,\mu_k\}$ be the set of all of the means of $G$ learned with the algorithm in the previous section.
These means are in the feature space $\mathbb{R}^p$.
Let $\gamma$ denote the current progress of the task, i.e., the previously achieved goal  + 1. 
We can define a reward function as follows:
\[
R(s,a) = -\|f(s,a) - \mu_{\gamma}\|_2^2 
\]

\subsection{Counter-examples}
We constructed two scenarios, a Maze and a Pinball domain, in which \hirl actually performs worse than the alternatives. In the Maze domain, there is a 2D grid with a single unique solution path to a goal state.
In this problem, the segments found by \hirl provide no additional information compared to IRL.
In the Pinball domain, there is a ball on a table with obstacles that is moved by tilting the table.
The ball has elastic collisions with the obstacles and has noisy dynamics.
In this domain, we find that \hirl tends to over-segment this problem since every collision results in another linear regime. 

\begin{table}[ht]
\scriptsize
    \centering
    \begin{tabular}{|r|r|r|r|r|}
    \hline
         %&  \multicolumn{2}{c|}{2D-MP-2}& %\multicolumn{2}{c|}{Two-Rooms} & 
         &\multicolumn{2}{c|}{Maze}& %\multicolumn{2}{c|}{RC(FO)}
         %& \multicolumn{2}{c|}{RC(PO)}& %\multicolumn{2}{c|}{Acrobot} & 
         \multicolumn{2}{c|}{Pinball}\\
    \hline
       & Max & AUC & Max & AUC\\
    \hline
        RL &  $\mathbf{0.960}$ & $2.575$& $0.481$ &$6.941$\\
    %\hline
    %    Sliding & $0.924$ & $-0.855$ & $0.242$ &$1.042$\\
    \hline
        IRL & $0.914$ & $\mathbf{3.575}$ & $0.424$ &$\mathbf{10.904}$\\
    \hline
        TSC+Endpoints & $0.944$ & $-0.448$&  $\mathbf{0.793}$ & $9.315$\\
    \hline
        \hirl & $0.924$ & $1.448$ & $0.722$ &$8.331$\\
    \hline
   % \hline
%        Perfect+Endpoints & - &-& $0.884$ & $34.236$\\
%    \hline
%        Perfect+\hirl & - & -& $0.934$ & $57.129$\\
%    \hline
    \end{tabular}
    \caption{This table summarizes the convergence rate and max reward attained by a Q-learning agent using different reward and state-space representations on domains that were constructed to be counter-examples. \hirl does not perform as well in domains where there is a single path to the goal state. In this case, IRL finds the path and the additional states added by \hirl can actually impede convergence. }
    \label{tab:my_label}
\end{table}

\end{document}