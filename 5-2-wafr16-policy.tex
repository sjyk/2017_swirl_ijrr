\section{\hirl: Policy Learning}
\seclabel{policy-learning}
If the demonstration dynamics are consistent with the execution setting, while learning the reward with MaxEnt-IRL, an optimal policy can simultaneously be extracted. However, we observed a number of practical challenges: (1) the reward function can often be represented more concisely than the policy, and as such, the demonstration data might be sufficient to estimate an accurate reward but not a useful policy, (2) the dynamics of the demonstration environment often differ slightly from the dynamics of the execution environment--making the rewards transferable but not the policies, and (3) transfer problems where the task instance has changed. 

Our solution is to refine the learned policy with physical rollouts. \hirl uses the learned transitions $[\rho_1,...,\rho_k]$ and $\mathbf{R}_{seq}$ as rewards for a Reinforcement Learning algorithm. In this section, we describe learning a policy $\pi$ given rewards $\mathbf{R}_{seq}$ and an ordered sequence of transitions $G$.
However, this problem is not trivial since solving $k$ independent problems neglects potential shared value structure between the local problems (e.g., a common failure state).
Furthermore, simply taking the aggregate of the rewards can lead to inconsistencies since there is nothing enforcing the order of operations.
The key insight is that a single policy can be learned jointly over all segments over a modified problem where the state-space with additional variables that keep track of the previously achieved segments.

\subsection{Off Policy RL Algorithms}


\subsubsection{Policy Learning}
Using our sequential task definition, we know that the reward transitions ($R_{i}$ to $R_{i+1}$) only depend on an arrival at the transition state $\rho_{i}$ and not any other aspect of the history.
Therefore, we can store a vector $v$, a $k$ dimensional binary vector ($v \in \{0,1\}^k$) that indicates whether a transition state $i \in 0,...,k$ has been reached.
This vector can be efficiently incremented when the current state $s \in \rho_{i+1}$.
Then, the additional complexity of representing the reward with history over $S \times  \{0,1\}^k$ is only $\mathcal{O}(k)$ instead of exponential in the time horizon.

The result is an augmented state-space $\binom{s}{v}$ to account for previous progress.
Over this state-space, we can apply Reinforcement Learning algorithms to iteratively converge to a successful policy for a new task instance.
\hirl applies Q-Learning with a Radial Basis Function value function representation to learn a policy $\pi$ over this state-space and the reward sequence $\mathbf{R}_{seq}$.
This is summarized in Algorithm~\ref{alg:tsh3}.


\begin{phase}[t]
\small
\DontPrintSemicolon
\caption{Policy Learning \label{alg:tsh3}}
\KwData{Transition States $G$, Reward Sequence $\mathbf{R}_{seq}$, exploration parameter $\epsilon$}

Initialize $Q(\binom{s}{v},a)$ randomly

\ForEach{$iter \in 0,...,I$}{
    Draw $s_0$ from initial conditions
    
    Initialize $v$ to be $[0,...,0]$
    
    Initialize $j$ to be $1$
    
    \ForEach{$t \in 0,...,T$}{
        Choose best action $a$ based on $Q$ or random action w.p $\epsilon$.
        
        Observe Reward $R_{j}$
        
        Update state to $s'$ and $Q$ via Q-Learning update
        
        If $s'$ is $\in  \rho_{j}$ update $v[j] = 1$ and $j = j +1$
    }
}

\KwResult{Policy $\pi$}
\end{phase}



\subsubsection{MDPs with Memory}
RL algorithms apply to problems that are specified as MDPs.
The challenge is that some sequential tasks may not be MDPs.
For example, attaining a reward at $\rho_i$ depends on knowing that the reward at goal $\rho_{i-1}$ was attained.
In general, to model this dependence on the past requires MDPs whose state-space also includes history.

Given a finite-horizon MDP $\mathcal{M}$ as defined in Section \ref{sec:back}, we can define an MDP $\mathcal{M}_H$ as follows.
Let $\mathcal{H}$ denote set of all dynamically feasible sequences of length smaller than $T$ comprised of the elements of $S$.
Therefore, for an agent at any time $t$, there is a sequence of previously visited states $H_t \in \mathcal{H}$.
The MDP $\mathcal{M}_H$ is defined as:
\[
\mathcal{M}_H = \langle S \times \mathcal{H},A,P'(\cdot,\cdot), R(\cdot,\cdot),T \rangle.
\]
For this MDP, $P'$ not only defines the transitions from the current state $s \mapsto s'$, but also increments the history sequence $H_{t+1} = H_{t} \sqcup s$.
Accordingly, the parametrized reward function $R$ is defined over $S$, $A$, and $H_{t+1}$.

$\mathcal{M}_H$ allows us to address the sequentiality problem since the reward is a function of the state and the history sequence.
However, without some parametrization of $H_t$, directly solving this MDPs with RL is impractical since it adds an overhead of $\mathcal{O}(e^{T})$ states.



